% @Author: YangZhou
% @Date:   2017-06-30 15:05:00
% @Last Modified by:   YangZhou
% @Last Modified time: 2018-01-16 23:36:43

\chapter{集中参数模型的POGO稳定性分析}
由于接下来我们希望研究strain对热导率的影响,为了合理地选取strain,我们必须知道这个结构在什么时候处于弹性拉伸区,什么时候会断。因此我们首先做一个拉伸测试。
左上和左下图表明无论strain有多小,stress都不为0,由于knot的存在必须有一个初始拉伸,所以0.0的位置Stress不为0。体系的稳定性依赖于一个外界strain,在这个范围内stress和strain成指数关系。且这个拉伸关系与温度关系不明显。
右上的图表是的是从0的位置向右加strain到0.05,然后向左strain变到-0.15,再回到0.0。这个图类似于磁滞回线,表面knot的状态与加压的历史有关,但是0.05左右的时候体系只有一个状态,这也就是我们要研究的strain。
右下图的是graphene相应的结果,可见knot与graphene的拉伸性质很不一样。

这里我们来研究knot的断裂。
左上图是同一个knot在不同初始seed情况下的多次拉伸结果。可以发现0.07以下的时候,体系处于一个确定的状态,与初始情况无关。这个strain以后可以认为knot第一次断裂,但是断得并不彻底,而是发生了塑性形变,如
红色的那一根线,先后断裂了3次才彻底断裂。各次断裂之间表现出线性关系,有确定的杨氏模量,但Stress的起伏也变大,起伏的周期也与相应的阶段有关。
左下图表面了体系中的应力分布,主要集中在结的附近,因此总是结最先断裂。
右上图表示用来打结的graphene的长度对这个关系的影响。越长的体系断裂得也越是彻底,最初断裂的位置变化不大,但有一定的位移。这个图的结果表明我们对不同lx的knot取0.05以下的strain是合理的。
右下图是左上图的各个曲线(共600次)对应平均。在第一次断裂以后Stress出现一个小幅增长,接着迅速下降,且之后随着strain的增加下降得越来越慢。更多的结果表明这个下降满足反比关系,即使strain=1.0,Stress也并非彻底为0.这确实是一个塑性形变。而grapheen的断裂相当干脆,是一种脆性的断裂。


ps,我刚刚想到对不同lx,把右下角的图头画一遍,从而不会那么空
左上图绿色表示的是第一次断裂的时候体系处于某个strain的概率,这个概率相当集中,基本上就在0.07附近,展宽很小。而红色的表示第二次断裂的时候的strain分布,它的最高点大概在0.17,展宽更大,所以这阶段的体系的稳定性不确定,可能在第一次断裂后不久就断裂了,也可能要过挺久。第三和第四次断裂的密度基本重合,且分布十分广泛。
左下图表示的是strain从0到1.0 变化的范围内体系断裂的次数的统计情况。可以发现,体系在这个范围内很可能断裂2到3次,但最多的断裂了9次之多。
右下图表示体系在这个strain下彻底断裂的可能性。可已看出,体系第一次断裂就很彻底的概率很小,但并非没有。到0.3附件,断裂的概率有了一半,到1.0附件,体系基本上已经全部断裂。
右上图是对断裂之前和两次断裂之间的体系的杨氏模量的统计。225左右的可能性最大的情况就是体系断裂前的杨氏模量。对与断裂后的情况,杨氏模量普遍变小,集中在150以下,在此之后可能性逐渐降低。因此断裂以后体系反倒更加容易拉伸了。
从这里开始我们研究这个材料的传热性质。先是长度和strain对热导率的影响。
左图表示不同strain情况下构成knot的graphene的长度对这个单结系统热导率的影响。在当前的尺度上,热导随长度基本是线性变化的。strain对热导率的影响不大。
右图显式地得到不同lx热导与strain的关系,可见strain对热导率的影响不大。相比较而言,graphene的strain对热导率影响挺大(然后引你的文章)
然后看看界面热导与温差的关系。再研究这个热导随superlattice长度的关系。
左上:对于特定的lx=70,kapitza与温差间的关系,可见dT(两端温差的一半)达到20K以后界面热阻不再变化。
左下:上图的每个点对应的温度分布
右上:用lx=70的graphene构成的knot去构造superlattice,nx表示superlattice包含几个结。计算的时候dT=100,横截面积取的相同的值,后面我会改成你建议的热流。这个图表明tc随nx基本上线性增加,在计算能承受的范围内没有要收敛的迹象。综合这个图与之前tc~lx的结果,我们得出,即使出现了knot这样的缺陷,在计算可承受的范围内热导依然不会收敛。
右下:温度分布与nx间的关系,横坐标是归一化的x。随nx的增大,温度分布逐渐趋向线性关系,knot构成的微观细节逐渐被抹去,在大尺度上可以被看做质点,于是可以被抽象为一个一维点阵,因此用一维点阵的结论可得tc发散。
右下inset:每个knot导致的温差delT*nx与nx的关系。在nx比较小的时候,knot处的温度下降占总温差的一半左右,而nx增大后,几乎全部的温差降落都是在knot处出现的,非knot处几乎没有温度降落。
这一页是还没完成的工作,我是感觉前面的东西太过流水账,没有太深刻的分析(不过我感觉这是我现在能做到的最好的程度了)。所以希望从谱的方面做一些补充。
左上:用速度关联函数求出每个原子的dos,然后对knot的bulk区域和结的区域分别计算dos,得到蓝色和绿色的线。红色的为graphene的dos。这个结果表明结的影响是非局域的,它让整个区域上的dos都偏离了graphene的结果,而不仅仅是knot的区域。基于这个部分,我打算计算一个很长的knot的dos,然后按离结的远近画更多区域的dos,看画到多远那个dos会像graphene的,如过足够远都不行,我们可以得出结论:结的存在是完全非局域的。而如果会像graphene的,那就更好了,就可以求出这个关联长度lc,进一步再研究strain,T什么的对它的影响,就又可以凑几个图。
右上:结的dos按xyz的分解,
右下:graphene的dos按xyz的分解,目前看来这两个图的意义不大。

所以我想改进点的地方就是,应该加点什么图,哪些图又该不要,应该在哪些地方多下点功夫,使得这个工作显得更有意义一些?


\bibliographystyle{FDUbib}
\bibliography{ref}