% !TEX encoding = UTF-8 Unicode
% XeLaTeX can use any Mac OS X font. See the setromanfont command below.
% Input to XeLaTeX is full Unicode, so Unicode characters can be typed directly into the source.

% The next lines tell TeXShop to typeset with xelatex, and to open and save the source with Unicode encoding.

%!TEX TS-program = xelatex

\documentclass[12pt]{article}
\usepackage[bottom=3cm]{geometry}                % See geometry.pdf to learn the layout options. There are lots.
\geometry{letterpaper}                   % ... or a4paper or a5paper or ... 
%\geometry{landscape}                % Activate for for rotated page geometry
%\usepackage[parfill]{parskip}    % Activate to begin paragraphs with an empty line rather than an indent
\usepackage{graphicx}
\usepackage{amssymb}
\usepackage{calc}
\usepackage{graphicx}
\pagestyle{empty}


%CJK Font Setup
\usepackage{xeCJK}
\punctstyle{kaiming}
%fc-list :lang=zh-cn to list all
\setCJKmainfont[BoldFont={NSimSun},ItalicFont={KaiTi}] {NSimSun}
\setCJKsansfont[BoldFont={SimHei}]{SimHei}
\setCJKmonofont[BoldFont={FangSong}]{FangSong}

% Will Robertson's fontspec.sty can be used to simplify font choices.
% To experiment, open /Applications/Font Book to examine the fonts provided on Mac OS X,
% and change "Hoefler Text" to any of these choices.

%English Font Setup
\usepackage{fontspec,xltxtra,xunicode}
\defaultfontfeatures{Mapping=tex-text}
\setmainfont{Times New Roman}
\setsansfont[Scale=MatchLowercase,Mapping=tex-text]{Calibri}
\setmonofont[Scale=MatchLowercase]{Courier New}


\author{Yang Zhou$<$\href{mailto:y_zhouy13@fudan.edu.cn}%
            {y_zhouy13@fudan.edu.cn}$>$}
%\date{}                                         % Activate to display a given date or no date

\newlength{\Han}
\settowidth{\Han}{汉}
\newcommand{\spreadCJK}[2]{\makebox[#1\Han][s]{#2}}

\begin{document}
\setlength{\headsep}{0cm}

\begin{flushright}
  \parbox[c]{6em}{ %
    {\small \makebox[\width][c]{学校代码:\quad10246 }\par
        \makebox[\width][c]{学\phantom{占位}号:\quad13110190020}}}
\end{flushright}
\vspace{\stretch{0.5}}

\begin{figure}[htbp]
  \begin{center}
    \includegraphics[width=0.5\textwidth]{images/FudanLOGO.eps}
  \end{center}
\end{figure}

\vspace{\stretch{0.5}}
\begin{center}
  {\Huge \makebox [0.45\textwidth][s]{博士学位论文}\par}
  {\Large \makebox [0.45\textwidth][s]{(学术学位)}\par}
  \vspace{\stretch{1.5}}
  {\LARGE \textsf{低维材料的热输运性质研究}\par}
  \vspace{\stretch{0.5}}
  {\Large \textrm{The thermal transport properties of low-dimensional materials research}\par}

  \vspace{\stretch{2}}
  \parbox[c]{0.5\textwidth}{%
    \large \setlength{\baselineskip}{1.5\baselineskip}%
    \spreadCJK{6}{院系}:\quad 物理学系\par
    \spreadCJK{6}{专业}:\quad 理论物理\par
    \spreadCJK{6}{姓名}:\quad 周阳\par
    \spreadCJK{6}{指导老师}:\quad 龚新高 \quad 教授\par
    \spreadCJK{6}{完成日期}:\quad 2017年4月1日}
\end{center}

\end{document}