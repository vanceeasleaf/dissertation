% @Author: YangZhou
% @Date:   2018-01-28 18:33:33
% @Last Modified by:   YangZhou
% @Last Modified time: 2018-01-28 18:35:08
\backchapter{致谢}

历经二十多年寒窗,把最美好的青春留在了校园,是的,他们就是生活工作在你我身边的博士。对于女博士,求学生涯,尤为不易。有快乐也有忧伤,有辛勤也有迷茫,有拼搏也有放弃,有收获也有缺憾。。。。。一朝功成之时,除了对身边人的致谢和感恩,更多的是对自我生命旅程的追忆和检醒。我们精选了几位女博士的毕业论文致谢。每个名字,都有许多难忘的瞬间;每句感谢,都是发自内心的真情流露。让我们一起怀念属于每个人的校园旅程。

首先,衷心感谢我的导师王怀民老师!从2000 年攻读硕士学位我就有幸成为了王老师的学生,在工作多年之后又得以继续在王老师的指导下完成博士学位的学习,在我人生的各个重要阶段王老师都给予我悉心指导和无私关怀。在课题选择和问题解决过程中,王老师以敏锐的学术洞察力和深厚的科研经验,高屋建瓴地为我论证把关。王老师在百忙之中坚持每周与我们进行学术讨论,每次讨论中,王老师总能对我繁乱的思绪抽丝剥茧,发现其中的闪光点并为我指明前进的方向。没有王老师的指导和帮助,我的研究工作将不可能顺利完成。王老师对我的言传身教将使我终身受益,他严谨的治学作风、高深的学术造诣、勇攀高峰的精神以及对科研工作的满腔热情,无时无刻不在激励与鞭策着我,永远是我工作学习的榜样。感谢我的师母陈海燕老师,陈老师在生活中给予了我和整个师门的同学们无微不至的关怀,使我们组成紧密团结的大家庭,共同成长。

衷心感谢软件工程中心的吴庆波主任、戴华东副主任、李佩江政委与孔金珠副主任!这些年来领导们始终对我的成长进步给予无私的关心与爱护,不仅给我读博深造的机会,还为我提供了宽松的研究环境,使我能够专注于课题研究。没有领导的支持与帮助,我的博士学业将不会有今天的进展。

衷心感谢尹刚老师与丁博老师!在整个博士课题研究过程中,两位老师始终给予我热情的指点与帮助,使我的研究工作能够顺利进行。老师们严谨的工作态度、深厚的理论功底、创新的研究思路,无不使我深感敬佩并始终将二位老师作为我学习的榜样。感谢史殿习老师、刘波老师、刘慧老师、唐扬斌老师、腾猛老师、李玺老师,老师们对待工作一丝不苟的敬业精神、团结协作严肃活泼的工作作风都使我深受感染,时刻感受到团队的力量。

感谢重点实验室的王戟老师、王意洁老师、彭宇行老师、李东升老师、陈振邦老师、褚瑞老师、张一鸣老师、黄震老师,老师们深厚的学术功底、丰富的科研经验、创新的研究思路,让我受益匪浅,为我在学术研究方面树立了很好的榜样。

感谢973 课题组的吕荣聪老师、徐洁老师、胡春明老师、蔡华老师、曹东刚老师、周扬帆老师与郑子彬老师,在历次学术交流中老师们所展现出的深厚的理论功底、严谨求实的工作态度、开阔的学术视野以及勇攀学术高峰的精神都令我深感敬佩,成为我今后努力的目标。

